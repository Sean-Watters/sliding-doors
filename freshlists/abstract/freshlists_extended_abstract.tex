% easychair.tex,v 3.5 2017/03/15

\documentclass[a4paper]{easychair}
%\documentclass[EPiC]{easychair}
%\documentclass[EPiCempty]{easychair}
%\documentclass[debug]{easychair}
%\documentclass[verbose]{easychair}
%\documentclass[notimes]{easychair}
%\documentclass[withtimes]{easychair}
%\documentclass[a4paper]{easychair}
%\documentclass[letterpaper]{easychair}

\usepackage{doc}
\usepackage{array}
\usepackage{enumerate}

% Maths fonts and stuff
\usepackage{amsmath}
\usepackage{amssymb}
\usepackage{bbm}
\usepackage{mathabx}
%\usepackage[greek,english]{babel}
\usepackage{mathtools}

% theorem environment
\newtheorem{theorem}{Theorem}
\newtheorem{definition}{Definition}

% agda
\definecolor{AgdaBound}{HTML}{000000}
\definecolor{AgdaDatatype}{HTML}{000000}
\definecolor{AgdaInductiveConstructor}{HTML}{000000}
\newcommand{\AgdaFontStyle}[1]{\textsf{#1}}
\newcommand{\AgdaBoundFontStyle}[1]{\textit{#1}}
\newcommand{\AgdaFormat}[2]{#2}
\newcommand{\AgdaNoSpaceMath}[1]
    {\begingroup\thickmuskip=0mu\medmuskip=0mu#1\endgroup}
\newcommand{\AgdaBound}[1]
    {\AgdaNoSpaceMath{\textcolor{AgdaBound}{\AgdaBoundFontStyle{\AgdaFormat{#1}{#1}}}}}
\newcommand{\AgdaDatatype}[1]
    {\AgdaNoSpaceMath{\textcolor{AgdaDatatype}{\AgdaFontStyle{\AgdaFormat{#1}{{#1}}}}}}
\newcommand{\AgdaInductiveConstructor}[1]
    {\AgdaNoSpaceMath{\textcolor{AgdaInductiveConstructor}{\AgdaFontStyle{\AgdaFormat{#1}{{#1}}}}}}

\newcommand\bv[1]{{\AgdaBound{$#1$}}}
\newcommand\ty[1]{\AgdaDatatype{$#1$}}
\newcommand{\flistsymb}{\AgdaDatatype{FList}}
\newcommand{\flist}[2]{\ensuremath{\flistsymb(#1,#2)}}
\newcommand{\freshfor}{\AgdaDatatype{\#}}
\newcommand{\fresh}[3]{{#2~\freshfor_{#1}~#3}}

\newcommand\nil{\AgdaInductiveConstructor{nil}}
\newcommand\cons{\AgdaInductiveConstructor{cons}}
\newcommand\nilF{\AgdaInductiveConstructor{nil$_{\#}$}}
\newcommand\consF{\AgdaInductiveConstructor{cons$_{\#}$}}
\newcommand\mergesymb{\cup}

\newcommand\univ{\mathsf{Type}} % the universe of sets
\newcommand\setuniv{\mathsf{Set}} % the universe of sets
\newcommand\propuniv{\mathsf{Prop}} % the universe of propositions
\newcommand\sto{\mathsf{STO}}
\newcommand\oicm{\mathsf{OICMon}}
\newcommand{\Forget}{\mathcal{U}}
\newcommand\forgetoicm{\Forget : \oicm{} \to{} \sto{}}

% use this if you have a long article and want to create an index
% \usepackage{makeidx}

% In order to save space or manage large tables or figures in a
% landcape-like text, you can use the rotating and pdflscape
% packages. Uncomment the desired from the below.
%
% \usepackage{rotating}
% \usepackage{pdflscape}

\title{Extensional Finite Sets and Multisets in Type Theory
\footnote{
This is an extended abstract of a paper that was published at APLAS 2023~\cite{freshlists}.
All results are formalised in Agda and are available at:
\url{https://www.seanwatters.uk/agda/fresh-lists/}.
}}

\author{
Clemens Kupke
\and
Fredrik Nordvall Forsberg
\and
Sean Watters
}

\institute{
  University of Strathclyde, UK\\
  \email{\{clemens.kupke, fredrik.nordvall-forsberg, sean.watters\}@strath.ac.uk}
  % \email{sean.watters@strath.ac.uk}
 }

 \authorrunning{
   Kupke, Nordvall Forsberg and
   Watters}
\titlerunning{Extensional Finite Sets and Multisets in Agda}

\begin{document}

\maketitle


% \begin{abstract}

The type of lists is one of the most elementary inductive data types.
It has been studied and used extensively by computer scientists and programmers for decades.
Two conceptually similar structures are those of finite sets and multisets, which can be thought of as unordered analogues to lists.
However, capturing unordered structures in a data type while maintaining desirable properties such as decidable equality and the correct equational theory is challenging.

The usual approach to formalise unordered structures in mathematics
is to represent them as functions (with finite support): finite sets as $X \to 2$, and finite multisets as $X \to \mathbb{N}$, respectively.
However, these representations do not enjoy decidable equality, even if the underlying type $X$ does.

Meanwhile the approach taken in most programming languages is to pretend --- one uses a list (or another ordered structure for efficiency) internally, but hides it and any invariants behind a layer of abstraction provided by an API.
However, each set or multiset can then be represented by many different lists,
meaning that the equational theory might not be correct. This is a problem
in a dependently typed setting, where
having equality as a first-class type allows us to
distinguish between different representations of the same set.

% The analogous approach in dependent type theory is to encode these invariants in an equivalence relation on lists,
% and define finite sets and multisets as setoids of lists plus the appropriate equivalence relation~\cite{gilles2003setoids}.
% However, this merely side-steps the issue;
% we may still have two distinct lists which represent the same finite (multi)set.
% Thus, we are forced to work with the equivalence relation at all times instead of the identity type.

In the setting of homotopy type theory (HoTT)~\cite{hottbook}, we can use higher inductive types (HITs) to define the identities on an inductive type simultaneously with its elements.
This allows us to define a data type which enjoys both decidable equality and the right equational theory, as demonstrated by Choudhury and Fiore~\cite{choudhuryfiore2023freecommmon}.
However, many proof assistants today do not support HITs;
thus, the main question we set out to answer in this work is
whether it is possible in ordinary dependent type theory
to define data types of finite sets and multisets, which:
\begin{enumerate}[(i)]
  \item have decidable equality iff the underlying set has decidable equality; and\label{item:success1}
  \item satisfy the equational theories of finite sets and multisets.\label{item:success2}
\end{enumerate}

For property~\eqref{item:success2}, we take as our success criteria the facts that
the type of finite sets is the free idempotent commutative monoid~\cite{frumin2018finsetshott},
and that finite multisets are the free commutative monoid.
Thus, we are really aiming to find data types for the free idempotent commutative monoid and free commutative monoid, which satisfy the above property~\eqref{item:success1}.
We accomplish this by restricting our attention to only those sets with decidable equality that can be totally ordered.
We can then form a type of sorted lists over such a set.
Provided we treat the existence of the ordering data carefully, this type turns out to give us exactly finite sets when the order is strict, and finite multisets when it is non-strict.


We show that our constructions satisfy universal properties, in the sense that they are left adjoints to forgetful functors --- this is the standard way to state freeness in the language of category theory.
However, note that the notion of freeness is with respect to e.g.\ totally ordered monoids, rather than all monoids.
%
For proving the universal properties and for defining the categories involved, we need function extensionality.
However we stress that the constructions themselves work in ordinary dependent type theory, without function extensionality.

\paragraph{Fresh Lists}
Fresh lists, the key inductive data type of this work, were first introduced by C.~Coquand to represent contexts in the simply typed lambda calculus~\cite{ccoquand2002formalstlc}.
The type of fresh lists is a parameterised data type similar to the type of ordinary lists,
with the additional requirement that in order to adjoin
a new element $\bv{x}$ to a list $\bv{xs}$, that element $\bv{x}$ must be ``fresh'' with respect to all other elements already present in the list $\bv{xs}$.
We follow the Agda standard library~\cite{agda-stdlib} in considering a generalised notion of freshness, given by an arbitrary binary relation on the carrier set.
We can recover Coquand's original notion of freshness by choosing inequality as our freshness relation.

%\begin{definition}
    %Given a type \ty{A} and a binary relation $\ty{R} : \ty{A} \to \ty{A} \to \univ$, we mutually inductively define a type $\flist{A}{R}$, together with a relation $\freshfor_R : A \to \flist{A}{R} \to \univ$, by the following constructors:
  %\begin{align*}
    %\nil &: \flist{A}{R} \\
      %\cons &: (\bv{x} : \ty{A}) \to (\bv{xs} : \flist{A}{R}) \to \bv{x}~\freshfor_R~\bv{xs} \to \flist{A}{R} \\[0.5em]
      %\nilF &: \{\bv{a} : \ty{A}\} \to \bv{a}~\freshfor_R~\nil \\
    %\consF &: \{\bv{a} : \ty{A}\} \to \{\bv{x} : \ty{A}\} \to \{\bv{xs} : \flist{A}{R}\} \to \{\bv{p} : \bv{x}~\freshfor_R~\bv{xs}\} \to \\
    %& \qquad\qquad\qquad\qquad\qquad\qquad \ty{R}~\bv{a}~\bv{x} \to \bv{a}~\freshfor_R~\bv{xs} \to \bv{a}~\freshfor_R~(\cons~\bv{x}~\bv{xs}~\bv{p}~)
  %\end{align*}
%For $a,x : A$, and  $xs : \flist{A}{R}$, we say that $a$ is fresh for $x$ when we have $R~a~x$, and that $a$ is fresh for $xs$ when we have $a~\freshfor_R~xs$.
%\end{definition}


\paragraph{Finite Sets as Sorted Lists}
Our candidate representation for finite sets satisfying the above properties~\eqref{item:success1} and~\eqref{item:success2} is the type of sorted lists without duplicates.
We obtain this by the appropriate instantiation of the type of fresh lists;
namely, \flist{A}{<} for some type $A : \setuniv$ and a strict total order $< \  : A \to A \to \propuniv$.
We then prove an extensionality principle analogous to set extensionality which allows us to show that \flist{A}{<} is an idempotent commutative monoid
with the empty list as the unit,
and the operation which merges two sorted lists as the multiplication.

%\begin{theorem}
%Given sorted lists $xs,ys : \flist{A}{<}$, we have $(a \in xs) \longleftrightarrow (a \in ys)$ for all $a : A$ iff $xs = ys$.
%\qed{}
%\end{theorem}

To establish~\eqref{item:success2}, we would like to show that this type is the \emph{free} idempotent commutative monoid.
However, there is a wrinkle --- the domain of the sorted list functor cannot be simply the category of sets $\setuniv$, since we require that the underlying set is equipped with a strict total order in order to form the type of sorted lists.
% since we require the data provided by the ordering to implement the monoid multiplication.
Assuming that any set can be equipped with such an order is a strongly classical axiom called the Ordering Principle which is strictly weaker than the well-ordering principle~\cite[Ch. 5 \S 5]{jech1973choice}, but still implies LEM~\cite{swanOPLEM}.
Therefore to remain constructive, we must restrict the domain of the functor to strictly totally ordered sets.
%
Thus, we define the categories $\sto$ of strictly totally ordered sets, and
$\oicm$ of \emph{ordered} idempotent commutative monoids (ordering data is also required for the monoids so that it can be preserved by the forgetful functor; this is satisfied for \flist{A}{<} via the lexicographic ordering).
With the categories in place, we can prove that the type of sorted lists is functorial, and left adjoint to the forgetful functor $\forgetoicm$, giving us the desired universal property.

\paragraph{Other Free Algebraic Structures}

The choice to implement sorted lists as an instantiation of the type of fresh lists reveals further paths to explore;
what happens for other instantiations of the freshness relation?
It turns out that different choices each yield a different free structure.

In particular, it should come as no surprise that finite multisets are represented by sorted lists with duplicates (i.e., fresh lists over a total order $\leq$).
The proof of the adjunction is very similar to the previous case,
however we obtain a different extensionality principle:
since the membership relation for multisets is valued in $\setuniv$ rather than $\propuniv$, we must prove an isomorphism rather than merely a bi-implication.
%
Other such results are summarised in Table~\ref{tab:other-results}.

\begin{table}[bh]\centering
  \begin{tabular}{  |c|m{16em}|m{12em}| }
    \hline
    Freshness Relation & \centering{Free Algebraic Structure} & \centering{Data Structure} \tabularnewline
                                                                \hline
                                                                $\leq$, a total order & Ordered Commutative Monoid & Sorted lists \\
    $<$, a strict total order & Ordered Idempotent Comm.\ Monoid & Sorted lists w/o duplicates \\
    $\lambda x. \lambda y. \bot$ & Pointed Set & \AgdaDatatype{Maybe} \\
    $\lambda x. \lambda y. \top$ & Monoid & \AgdaDatatype{List} \\
    $\neq$ & Left-Regular Band Monoid & Lists without duplicates \\
    $=$ & Reflexive Partial Monoid & $1 + (A \times \mathbb{N}^{>0})$ \\
    \hline
  \end{tabular}
  \caption{Free algebraic structures as instantiations of freshlists (carrier set $A$)}
  \label{tab:other-results}
\end{table}

\paragraph{Related Work}
Appel and Leroy~\cite{appelleroy2023tries} recently introduced canonical binary tries as an extensional representation of finite \emph{maps}.
These can be used to construct finite sets with elements from the index type.
Krebbers~\cite{krebbers2023extensionalmaps} extended this technique to form extensional finite maps over arbitrary countable sets of keys.

The technique of using underlying ordering data to construct extensional data structures is not new, and has been employed in a number of Coq libraries for many years~\cite{coqextructures}\cite{coqfinmap}\cite{coqfset}\cite{coqssrmisc}.


% The use of tries allows for significantly improved lookup performance compared to lists, and with more work, it is conceivable that finite sets with elements from an arbitrary but fixed first-order data type could be extensionally represented this way~\cite{hinze2000tries}.
% Our representation using sorted lists is less efficient, but on the other hand works uniformly in the element type, as long as it is equipped with a total order.

% \end{abstract}


\bibliographystyle{plain}
%\bibliographystyle{alpha}
%\bibliographystyle{unsrt}
%\bibliographystyle{abbrv}
\bibliography{extabs}

%------------------------------------------------------------------------------
\end{document}

