\documentclass{beamer}
\usetheme{Boadilla}

\usepackage{tikz}
\usepackage{amsmath}
\usepackage{unicode-math}
\usepackage{mathtools}
\usepackage[conor]{agda}

\setmathfont{XITS Math}
%%%%%%%%%%
% latex preamble
% (missing unicode chars)

\usepackage{newunicodechar}
\newunicodechar{∈}{\ensuremath{\mathnormal{\in}}}
\newunicodechar{≡}{\ensuremath{\mathnormal{\equiv}}}
\newunicodechar{∷}{\ensuremath{\mathnormal{\dblcolon}}}
\newunicodechar{⊤}{\ensuremath{\mathnormal{\top}}}
\newunicodechar{₀}{\ensuremath{\mathnormal{_0}}}
\newunicodechar{₁}{\ensuremath{\mathnormal{_1}}}
\newunicodechar{₂}{\ensuremath{\mathnormal{_2}}}
\newunicodechar{π}{\ensuremath{\mathnormal{π}}}
\newunicodechar{∀}{\ensuremath{\mathnormal{\forall}}}
\newunicodechar{ℕ}{\ensuremath{\mathbb{N}}}
\newunicodechar{μ}{\ensuremath{\mathnormal{\mu}}}
\newunicodechar{ϕ}{\ensuremath{\mathnormal{\varphi}}}
\newunicodechar{ψ}{\ensuremath{\mathnormal{\psi}}}
\newunicodechar{η}{\ensuremath{\mathnormal{\eta}}}
\newunicodechar{≈}{\ensuremath{\mathnormal{\approx}}}

%%%%%%%%%%
% agda preamble

\begin{code}[hide]%
\>[0]\<%
\\
\>[0]\AgdaKeyword{open}\AgdaSpace{}%
\AgdaKeyword{import}\AgdaSpace{}%
\AgdaModule{Data.Nat}\AgdaSpace{}%
\AgdaKeyword{hiding}\AgdaSpace{}%
\AgdaSymbol{(}\AgdaOperator{\AgdaFunction{\AgdaUnderscore{}≟\AgdaUnderscore{}}}\AgdaSymbol{)}\<%
\\
\>[0]\AgdaKeyword{open}\AgdaSpace{}%
\AgdaKeyword{import}\AgdaSpace{}%
\AgdaModule{Data.Fin}\AgdaSpace{}%
\AgdaKeyword{using}\AgdaSpace{}%
\AgdaSymbol{(}\AgdaDatatype{Fin}\AgdaSymbol{;}\AgdaSpace{}%
\AgdaInductiveConstructor{zero}\AgdaSymbol{;}\AgdaSpace{}%
\AgdaInductiveConstructor{suc}\AgdaSymbol{;}\AgdaSpace{}%
\AgdaOperator{\AgdaFunction{\AgdaUnderscore{}≟\AgdaUnderscore{}}}\AgdaSymbol{)}\AgdaSpace{}%
\AgdaKeyword{renaming}\AgdaSpace{}%
\AgdaSymbol{(}\AgdaFunction{inject₁}\AgdaSpace{}%
\AgdaSymbol{to}\AgdaSpace{}%
\AgdaFunction{fin-inject₁}\AgdaSymbol{)}\<%
\\
\>[0]\AgdaKeyword{open}\AgdaSpace{}%
\AgdaKeyword{import}\AgdaSpace{}%
\AgdaModule{Data.Product}\<%
\\
\>[0]\AgdaKeyword{open}\AgdaSpace{}%
\AgdaKeyword{import}\AgdaSpace{}%
\AgdaModule{Relation.Binary.PropositionalEquality}\<%
\\
\>[0]\AgdaKeyword{open}\AgdaSpace{}%
\AgdaKeyword{import}\AgdaSpace{}%
\AgdaModule{Relation.Nullary.Decidable}\<%
\\
%
\\[\AgdaEmptyExtraSkip]%
\>[0]\AgdaKeyword{data}\AgdaSpace{}%
\AgdaDatatype{Opη}\AgdaSpace{}%
\AgdaSymbol{:}\AgdaSpace{}%
\AgdaPrimitive{Set}\AgdaSpace{}%
\AgdaKeyword{where}\<%
\\
\>[0][@{}l@{\AgdaIndent{0}}]%
\>[2]\AgdaInductiveConstructor{mu}\AgdaSpace{}%
\AgdaInductiveConstructor{nu}\AgdaSpace{}%
\AgdaSymbol{:}\AgdaSpace{}%
\AgdaDatatype{Opη}\<%
\\
%
\\[\AgdaEmptyExtraSkip]%
\>[0]\AgdaKeyword{data}\AgdaSpace{}%
\AgdaDatatype{Op₀}\AgdaSpace{}%
\AgdaSymbol{(}\AgdaBound{At}\AgdaSpace{}%
\AgdaSymbol{:}\AgdaSpace{}%
\AgdaPrimitive{Set}\AgdaSymbol{)}\AgdaSpace{}%
\AgdaSymbol{:}\AgdaSpace{}%
\AgdaPrimitive{Set}\AgdaSpace{}%
\AgdaKeyword{where}\<%
\\
\>[0][@{}l@{\AgdaIndent{0}}]%
\>[2]\AgdaInductiveConstructor{tt}\AgdaSpace{}%
\AgdaInductiveConstructor{ff}\AgdaSpace{}%
\AgdaSymbol{:}\AgdaSpace{}%
\AgdaDatatype{Op₀}\AgdaSpace{}%
\AgdaBound{At}\<%
\\
%
\>[2]\AgdaInductiveConstructor{at}\AgdaSpace{}%
\AgdaInductiveConstructor{¬at}\AgdaSpace{}%
\AgdaSymbol{:}\AgdaSpace{}%
\AgdaBound{At}\AgdaSpace{}%
\AgdaSymbol{→}\AgdaSpace{}%
\AgdaDatatype{Op₀}\AgdaSpace{}%
\AgdaBound{At}\<%
\\
%
\\[\AgdaEmptyExtraSkip]%
\>[0]\AgdaKeyword{data}\AgdaSpace{}%
\AgdaDatatype{Op₁}\AgdaSpace{}%
\AgdaSymbol{:}\AgdaSpace{}%
\AgdaPrimitive{Set}\AgdaSpace{}%
\AgdaKeyword{where}\<%
\\
\>[0][@{}l@{\AgdaIndent{0}}]%
\>[2]\AgdaInductiveConstructor{box}\AgdaSpace{}%
\AgdaInductiveConstructor{dia}\AgdaSpace{}%
\AgdaSymbol{:}\AgdaSpace{}%
\AgdaDatatype{Op₁}\<%
\\
%
\\[\AgdaEmptyExtraSkip]%
\>[0]\AgdaKeyword{data}\AgdaSpace{}%
\AgdaDatatype{Op₂}\AgdaSpace{}%
\AgdaSymbol{:}\AgdaSpace{}%
\AgdaPrimitive{Set}\AgdaSpace{}%
\AgdaKeyword{where}\<%
\\
\>[0][@{}l@{\AgdaIndent{0}}]%
\>[2]\AgdaInductiveConstructor{and}\AgdaSpace{}%
\AgdaInductiveConstructor{or}\AgdaSpace{}%
\AgdaSymbol{:}\AgdaSpace{}%
\AgdaDatatype{Op₂}\<%
\\
\>[0]\<%
\end{code}

%%%%%%%%%%
% agda snippets

\newcommand{\snippetwellscoped}{%
\begin{code}%
\>[0]\AgdaKeyword{data}\AgdaSpace{}%
\AgdaDatatype{WST}\AgdaSpace{}%
\AgdaSymbol{(}\AgdaBound{At}\AgdaSpace{}%
\AgdaSymbol{:}\AgdaSpace{}%
\AgdaPrimitive{Set}\AgdaSymbol{)}\AgdaSpace{}%
\AgdaSymbol{(}\AgdaBound{n}\AgdaSpace{}%
\AgdaSymbol{:}\AgdaSpace{}%
\AgdaDatatype{ℕ}\AgdaSymbol{)}\AgdaSpace{}%
\AgdaSymbol{:}\AgdaSpace{}%
\AgdaPrimitive{Set}\AgdaSpace{}%
\AgdaKeyword{where}\<%
\\
\>[0][@{}l@{\AgdaIndent{0}}]%
\>[2]\AgdaInductiveConstructor{tt}\AgdaSpace{}%
\AgdaInductiveConstructor{ff}%
\>[13]\AgdaSymbol{:}\AgdaSpace{}%
\AgdaDatatype{WST}\AgdaSpace{}%
\AgdaBound{At}\AgdaSpace{}%
\AgdaBound{n}\<%
\\
%
\>[2]\AgdaInductiveConstructor{at}\AgdaSpace{}%
\AgdaInductiveConstructor{¬at}%
\>[13]\AgdaSymbol{:}\AgdaSpace{}%
\AgdaBound{At}\AgdaSpace{}%
\AgdaSymbol{→}\AgdaSpace{}%
\AgdaDatatype{WST}\AgdaSpace{}%
\AgdaBound{At}\AgdaSpace{}%
\AgdaBound{n}\<%
\\
%
\>[2]\AgdaInductiveConstructor{and}\AgdaSpace{}%
\AgdaInductiveConstructor{or}%
\>[13]\AgdaSymbol{:}\AgdaSpace{}%
\AgdaSymbol{(}\AgdaBound{ϕ}\AgdaSpace{}%
\AgdaBound{ψ}\AgdaSpace{}%
\AgdaSymbol{:}\AgdaSpace{}%
\AgdaDatatype{WST}\AgdaSpace{}%
\AgdaBound{At}\AgdaSpace{}%
\AgdaBound{n}\AgdaSymbol{)}\AgdaSpace{}%
\AgdaSymbol{→}\AgdaSpace{}%
\AgdaDatatype{WST}\AgdaSpace{}%
\AgdaBound{At}\AgdaSpace{}%
\AgdaBound{n}\<%
\\
%
\>[2]\AgdaInductiveConstructor{box}\AgdaSpace{}%
\AgdaInductiveConstructor{dia}\AgdaSpace{}%
\AgdaSymbol{:}\AgdaSpace{}%
\AgdaSymbol{(}\AgdaBound{ϕ}\AgdaSpace{}%
\AgdaSymbol{:}\AgdaSpace{}%
\AgdaDatatype{WST}\AgdaSpace{}%
\AgdaBound{At}\AgdaSpace{}%
\AgdaBound{n}\AgdaSymbol{)}\AgdaSpace{}%
\AgdaSymbol{→}\AgdaSpace{}%
\AgdaDatatype{WST}\AgdaSpace{}%
\AgdaBound{At}\AgdaSpace{}%
\AgdaBound{n}\<%
\\
%
\>[2]\AgdaInductiveConstructor{mu}\AgdaSpace{}%
\AgdaInductiveConstructor{nu}%
\>[13]\AgdaSymbol{:}\AgdaSpace{}%
\AgdaSymbol{(}\AgdaBound{ϕ}\AgdaSpace{}%
\AgdaSymbol{:}\AgdaSpace{}%
\AgdaDatatype{WST}\AgdaSpace{}%
\AgdaBound{At}\AgdaSpace{}%
\AgdaSymbol{(}\AgdaInductiveConstructor{suc}\AgdaSpace{}%
\AgdaBound{n}\AgdaSymbol{))}\AgdaSpace{}%
\AgdaSymbol{→}\AgdaSpace{}%
\AgdaDatatype{WST}\AgdaSpace{}%
\AgdaBound{At}\AgdaSpace{}%
\AgdaBound{n}\<%
\\
%
\>[2]\AgdaInductiveConstructor{var}%
\>[13]\AgdaSymbol{:}\AgdaSpace{}%
\AgdaDatatype{Fin}\AgdaSpace{}%
\AgdaBound{n}\AgdaSpace{}%
\AgdaSymbol{→}\AgdaSpace{}%
\AgdaDatatype{WST}\AgdaSpace{}%
\AgdaBound{At}\AgdaSpace{}%
\AgdaBound{n}\<%
\end{code}}

\begin{code}[hide]%
\>[0]\AgdaKeyword{data}\AgdaSpace{}%
\AgdaDatatype{IsFP}\AgdaSpace{}%
\AgdaSymbol{\{}\AgdaBound{At}\AgdaSpace{}%
\AgdaSymbol{:}\AgdaSpace{}%
\AgdaPrimitive{Set}\AgdaSymbol{\}}\AgdaSpace{}%
\AgdaSymbol{\{}\AgdaBound{n}\AgdaSpace{}%
\AgdaSymbol{:}\AgdaSpace{}%
\AgdaDatatype{ℕ}\AgdaSymbol{\}}\AgdaSpace{}%
\AgdaSymbol{:}\AgdaSpace{}%
\AgdaDatatype{WST}\AgdaSpace{}%
\AgdaBound{At}\AgdaSpace{}%
\AgdaBound{n}\AgdaSpace{}%
\AgdaSymbol{→}\AgdaSpace{}%
\AgdaPrimitive{Set}\AgdaSpace{}%
\AgdaKeyword{where}\<%
\\
\>[0][@{}l@{\AgdaIndent{0}}]%
\>[2]\AgdaInductiveConstructor{mu}\AgdaSpace{}%
\AgdaSymbol{:}\AgdaSpace{}%
\AgdaSymbol{(}\AgdaBound{ϕ}\AgdaSpace{}%
\AgdaSymbol{:}\AgdaSpace{}%
\AgdaDatatype{WST}\AgdaSpace{}%
\AgdaBound{At}\AgdaSpace{}%
\AgdaSymbol{(}\AgdaInductiveConstructor{suc}\AgdaSpace{}%
\AgdaBound{n}\AgdaSymbol{))}\AgdaSpace{}%
\AgdaSymbol{→}\AgdaSpace{}%
\AgdaDatatype{IsFP}\AgdaSpace{}%
\AgdaSymbol{(}\AgdaInductiveConstructor{mu}\AgdaSpace{}%
\AgdaBound{ϕ}\AgdaSymbol{)}\<%
\\
%
\>[2]\AgdaInductiveConstructor{nu}\AgdaSpace{}%
\AgdaSymbol{:}\AgdaSpace{}%
\AgdaSymbol{(}\AgdaBound{ϕ}\AgdaSpace{}%
\AgdaSymbol{:}\AgdaSpace{}%
\AgdaDatatype{WST}\AgdaSpace{}%
\AgdaBound{At}\AgdaSpace{}%
\AgdaSymbol{(}\AgdaInductiveConstructor{suc}\AgdaSpace{}%
\AgdaBound{n}\AgdaSymbol{))}\AgdaSpace{}%
\AgdaSymbol{→}\AgdaSpace{}%
\AgdaDatatype{IsFP}\AgdaSpace{}%
\AgdaSymbol{(}\AgdaInductiveConstructor{nu}\AgdaSpace{}%
\AgdaBound{ϕ}\AgdaSymbol{)}\<%
\end{code}

\newcommand{\snippetscope}{%
\begin{code}%
\>[0]\AgdaKeyword{data}\AgdaSpace{}%
\AgdaDatatype{Scope}\AgdaSpace{}%
\AgdaSymbol{(}\AgdaBound{At}\AgdaSpace{}%
\AgdaSymbol{:}\AgdaSpace{}%
\AgdaPrimitive{Set}\AgdaSymbol{)}\AgdaSpace{}%
\AgdaSymbol{:}\AgdaSpace{}%
\AgdaDatatype{ℕ}\AgdaSpace{}%
\AgdaSymbol{→}\AgdaSpace{}%
\AgdaPrimitive{Set}\AgdaSpace{}%
\AgdaKeyword{where}\<%
\\
\>[0][@{}l@{\AgdaIndent{0}}]%
\>[2]\AgdaInductiveConstructor{[]}\AgdaSpace{}%
\AgdaSymbol{:}\AgdaSpace{}%
\AgdaDatatype{Scope}\AgdaSpace{}%
\AgdaBound{At}\AgdaSpace{}%
\AgdaInductiveConstructor{zero}\<%
\\
%
\>[2]\AgdaOperator{\AgdaInductiveConstructor{\AgdaUnderscore{}-,\AgdaUnderscore{}}}%
\>[167I]\AgdaSymbol{:}\AgdaSpace{}%
\AgdaSymbol{∀}\AgdaSpace{}%
\AgdaSymbol{\{}\AgdaBound{n}\AgdaSymbol{\}}\AgdaSpace{}%
\AgdaSymbol{(}\AgdaBound{Γ₀}\AgdaSpace{}%
\AgdaSymbol{:}\AgdaSpace{}%
\AgdaDatatype{Scope}\AgdaSpace{}%
\AgdaBound{At}\AgdaSpace{}%
\AgdaBound{n}\AgdaSymbol{)}\AgdaSpace{}%
\AgdaSymbol{\{}\AgdaBound{ϕ}\AgdaSpace{}%
\AgdaSymbol{:}\AgdaSpace{}%
\AgdaDatatype{WST}\AgdaSpace{}%
\AgdaBound{At}\AgdaSpace{}%
\AgdaBound{n}\AgdaSymbol{\}}\<%
\\
\>[.][@{}l@{}]\<[167I]%
\>[7]\AgdaSymbol{→}\AgdaSpace{}%
\AgdaSymbol{(}\AgdaBound{Γ₀}\AgdaSpace{}%
\AgdaSymbol{:}\AgdaSpace{}%
\AgdaDatatype{IsFP}\AgdaSpace{}%
\AgdaBound{ϕ}\AgdaSymbol{)}\AgdaSpace{}%
\AgdaSymbol{→}\AgdaSpace{}%
\AgdaDatatype{Scope}\AgdaSpace{}%
\AgdaBound{At}\AgdaSpace{}%
\AgdaSymbol{(}\AgdaInductiveConstructor{suc}\AgdaSpace{}%
\AgdaBound{n}\AgdaSymbol{)}\<%
\end{code}}

\newcommand{\snippetsublimelyscoped}{%
\begin{code}%
\>[0]\AgdaKeyword{mutual}\<%
\\
\>[0][@{}l@{\AgdaIndent{0}}]%
\>[2]\AgdaKeyword{data}\AgdaSpace{}%
\AgdaDatatype{SST}\AgdaSpace{}%
\AgdaSymbol{(}\AgdaBound{At}\AgdaSpace{}%
\AgdaSymbol{:}\AgdaSpace{}%
\AgdaPrimitive{Set}\AgdaSymbol{)}\AgdaSpace{}%
\AgdaSymbol{\{}\AgdaBound{n}\AgdaSpace{}%
\AgdaSymbol{:}\AgdaSpace{}%
\AgdaDatatype{ℕ}\AgdaSymbol{\}}\AgdaSpace{}%
\AgdaSymbol{(}\AgdaBound{Γ}\AgdaSpace{}%
\AgdaSymbol{:}\AgdaSpace{}%
\AgdaDatatype{Scope}\AgdaSpace{}%
\AgdaBound{At}\AgdaSpace{}%
\AgdaBound{n}\AgdaSymbol{)}\AgdaSpace{}%
\AgdaSymbol{:}\AgdaSpace{}%
\AgdaPrimitive{Set}\AgdaSpace{}%
\AgdaKeyword{where}\<%
\\
\>[2][@{}l@{\AgdaIndent{0}}]%
\>[4]\AgdaComment{--\ other\ constructors\ here...}\<%
\\
%
\>[4]\AgdaInductiveConstructor{var}\AgdaSpace{}%
\AgdaSymbol{:}\AgdaSpace{}%
\AgdaSymbol{(}\AgdaBound{x}\AgdaSpace{}%
\AgdaSymbol{:}\AgdaSpace{}%
\AgdaDatatype{Fin}\AgdaSpace{}%
\AgdaBound{n}\AgdaSymbol{)}\AgdaSpace{}%
\AgdaSymbol{→}\AgdaSpace{}%
\AgdaDatatype{SST}\AgdaSpace{}%
\AgdaBound{At}\AgdaSpace{}%
\AgdaBound{Γ}\<%
\\
%
\>[4]\AgdaInductiveConstructor{mu}%
\>[8]\AgdaSymbol{:}\AgdaSpace{}%
\AgdaSymbol{\{}\AgdaBound{ψ}\AgdaSpace{}%
\AgdaSymbol{:}\AgdaSpace{}%
\AgdaDatatype{WST}\AgdaSpace{}%
\AgdaBound{At}\AgdaSpace{}%
\AgdaSymbol{(}\AgdaInductiveConstructor{suc}\AgdaSpace{}%
\AgdaBound{n}\AgdaSymbol{)\}}\<%
\\
%
\>[8]\AgdaSymbol{→}\AgdaSpace{}%
\AgdaSymbol{(}\AgdaBound{ϕ}\AgdaSpace{}%
\AgdaSymbol{:}\AgdaSpace{}%
\AgdaDatatype{SST}\AgdaSpace{}%
\AgdaBound{At}\AgdaSpace{}%
\AgdaSymbol{(}\AgdaBound{Γ}\AgdaSpace{}%
\AgdaOperator{\AgdaInductiveConstructor{-,}}\AgdaSpace{}%
\AgdaInductiveConstructor{mu}\AgdaSpace{}%
\AgdaBound{ψ}\AgdaSymbol{))}\<%
\\
%
\>[8]\AgdaSymbol{→}\AgdaSpace{}%
\AgdaBound{ψ}\AgdaSpace{}%
\AgdaOperator{\AgdaDatatype{≈}}\AgdaSpace{}%
\AgdaBound{ϕ}\<%
\\
%
\>[8]\AgdaSymbol{→}\AgdaSpace{}%
\AgdaDatatype{SST}\AgdaSpace{}%
\AgdaBound{At}\AgdaSpace{}%
\AgdaBound{Γ}\<%
\\
%
\\[\AgdaEmptyExtraSkip]%
%
\>[2]\AgdaKeyword{data}\AgdaSpace{}%
\AgdaOperator{\AgdaDatatype{\AgdaUnderscore{}≈\AgdaUnderscore{}}}\AgdaSpace{}%
\AgdaSymbol{\{}\AgdaBound{At}\AgdaSpace{}%
\AgdaSymbol{:}\AgdaSpace{}%
\AgdaPrimitive{Set}\AgdaSymbol{\}}\AgdaSpace{}%
\AgdaSymbol{\{}\AgdaBound{n}\AgdaSpace{}%
\AgdaSymbol{:}\AgdaSpace{}%
\AgdaDatatype{ℕ}\AgdaSymbol{\}}\AgdaSpace{}%
\AgdaSymbol{\{}\AgdaBound{Γ}\AgdaSpace{}%
\AgdaSymbol{:}\AgdaSpace{}%
\AgdaDatatype{Scope}\AgdaSpace{}%
\AgdaBound{At}\AgdaSpace{}%
\AgdaBound{n}\AgdaSymbol{\}}\<%
\\
\>[2][@{}l@{\AgdaIndent{0}}]%
\>[4]\AgdaSymbol{:}\AgdaSpace{}%
\AgdaDatatype{WST}\AgdaSpace{}%
\AgdaBound{At}\AgdaSpace{}%
\AgdaBound{n}\AgdaSpace{}%
\AgdaSymbol{→}\AgdaSpace{}%
\AgdaDatatype{SST}\AgdaSpace{}%
\AgdaBound{At}\AgdaSpace{}%
\AgdaBound{Γ}\AgdaSpace{}%
\AgdaSymbol{→}\AgdaSpace{}%
\AgdaPrimitive{Set}\AgdaSpace{}%
\AgdaKeyword{where}\<%
\\
%
\>[4]\AgdaComment{--\ other\ constructors\ here...}\<%
\\
%
\>[4]\AgdaInductiveConstructor{var}\AgdaSpace{}%
\AgdaSymbol{:}\AgdaSpace{}%
\AgdaSymbol{(}\AgdaBound{x}\AgdaSpace{}%
\AgdaSymbol{:}\AgdaSpace{}%
\AgdaDatatype{Fin}\AgdaSpace{}%
\AgdaBound{n}\AgdaSymbol{)}\AgdaSpace{}%
\AgdaSymbol{→}\AgdaSpace{}%
\AgdaSymbol{(}\AgdaInductiveConstructor{var}\AgdaSpace{}%
\AgdaBound{x}\AgdaSymbol{)}\AgdaSpace{}%
\AgdaOperator{\AgdaDatatype{≈}}\AgdaSpace{}%
\AgdaSymbol{(}\AgdaInductiveConstructor{var}\AgdaSpace{}%
\AgdaBound{x}\AgdaSymbol{)}\<%
\\
%
\>[4]\AgdaInductiveConstructor{mu}%
\>[8]\AgdaSymbol{:}\AgdaSpace{}%
\AgdaSymbol{\{}\AgdaBound{ϕ}\AgdaSpace{}%
\AgdaSymbol{:}\AgdaSpace{}%
\AgdaDatatype{WST}\AgdaSpace{}%
\AgdaBound{At}\AgdaSpace{}%
\AgdaSymbol{(}\AgdaInductiveConstructor{suc}\AgdaSpace{}%
\AgdaBound{n}\AgdaSymbol{)\}}\<%
\\
%
\>[8]\AgdaSymbol{→}\AgdaSpace{}%
\AgdaSymbol{\{}\AgdaBound{ϕ'}\AgdaSpace{}%
\AgdaSymbol{:}\AgdaSpace{}%
\AgdaDatatype{SST}\AgdaSpace{}%
\AgdaBound{At}\AgdaSpace{}%
\AgdaSymbol{(}\AgdaBound{Γ}\AgdaSpace{}%
\AgdaOperator{\AgdaInductiveConstructor{-,}}\AgdaSpace{}%
\AgdaInductiveConstructor{mu}\AgdaSpace{}%
\AgdaBound{ϕ}\AgdaSymbol{)\}}\<%
\\
%
\>[8]\AgdaSymbol{→}\AgdaSpace{}%
\AgdaSymbol{(}\AgdaBound{p}\AgdaSpace{}%
\AgdaSymbol{:}\AgdaSpace{}%
\AgdaBound{ϕ}\AgdaSpace{}%
\AgdaOperator{\AgdaDatatype{≈}}\AgdaSpace{}%
\AgdaBound{ϕ'}\AgdaSymbol{)}\<%
\\
%
\>[8]\AgdaSymbol{→}\AgdaSpace{}%
\AgdaInductiveConstructor{mu}\AgdaSpace{}%
\AgdaBound{ϕ}\AgdaSpace{}%
\AgdaOperator{\AgdaDatatype{≈}}\AgdaSpace{}%
\AgdaInductiveConstructor{mu}\AgdaSpace{}%
\AgdaBound{ϕ'}\AgdaSpace{}%
\AgdaBound{p}\<%
\end{code}}
 % agda code snippets

% manual agdfa typesetting
\newcommand{\xvar}{\AgdaBound{$x$}}
\newcommand{\yvar}{\AgdaBound{$y$}}
\newcommand{\nil}{\AgdaField{$[]$}}
\newcommand{\cons}{\AgdaField{$\dblcolon$}}
\newcommand{\phitm}{\AgdaFunction{$\phi$}}
\newcommand{\datatm}{\AgdaDatatype{Tm}}
\newcommand{\abstm}{\AgdaField{abs}}
\newcommand{\union}{\AgdaFunction{$\cup$}}

\title[Extenensional Fin. Sets \& Multisets in TT]{Extensional Finite Sets and Multisets in Type Theory}
% \subtitle{Free Algebraic Structures via Fresh Lists}
\author[S. Watters]{Clemens Kupke \and Fredrik Nordvall Forsberg \and \underline{\textbf{Sean Watters}}}
\institute{University of Strathclyde}
\date{13/06/2024}

\begin{document}


\begin{frame}
  \titlepage{}
\end{frame}

\begin{frame}{Motivation}

  We want a data type for collections of unordered data (ie, finite sets and multisets), which:
  \begin{itemize}
    \item Has decidable equality iff the underlying type does.
    \item Satisfies the expected equational theories.
  \end{itemize}

  \begin{block}{Our Contribution}
  \begin{itemize}
    \item We show that a suitable generalisation of Catarina Coquand's data type of fresh lists acheives the above, and also realises many other free algebraic structures.

    \item In each case, we find a free-forgetful adjunction.
  \end{itemize}
  \end{block}

\end{frame}

% \begin{frame}{Motivation: A Normal Form for Contexts}
% \begin{exampleblock}{A Syntax with Binding}
% \snippetdatatm{}
% \end{exampleblock}
% \pause

% \begin{itemize}
%   \item  Consider the term $\abstm~\xvar~(\abstm~\yvar~\AgdaHole{?})$.
%         \pause

%   \item  Notice that the hole has type $\datatm~(\yvar~\cons~\xvar~\cons~\nil)$.
%         \pause

% \item  This means if we had some $\phitm : \datatm~(\xvar~\cons~\yvar~\cons~\nil)$, it would not fit!
% \end{itemize}
% \pause

% \begin{block}{Goal}
%   If we need such a $\phi$ to fit in such a hole, then we need contexts that are \emph{sets}, not lists.
% \end{block}
% \end{frame}


\begin{frame}{Notions of Subsets and Multisets in Type Theory}
\begin{block}{In Short}
\begin{itemize}
  \item Given some $S : \mathsf{Set}$, subsets of $S$ are unary predicates $S \to \mathsf{Prop}$.
  \item \emph{Decidable} subsets are functions $S \to 2$.
  \item Multisets over $S$ are functions $S \to \mathbb{N}$.
\end{itemize}
\end{block}
\pause

\begin{block}{Desirable Properties}
\begin{itemize}
  \item Extenensionality: \pause $(X = Y) \iff (\forall x.~x \in X \iff x \in Y)$
  \item Decidable Equality \pause $\Leftarrow$ Finiteness
\end{itemize}
\end{block}
\end{frame}

\begin{frame}{Finiteness, Decidable Equality, Extensionality}
\begin{block}{Some Attempts at Respresenting Finite Sets}
\begin{itemize}
  \item  $S \to 2$? \\
        (No way to enforce finiteness; no dec. eq.)
  \item An enumeration list? \\
        (No extensionality.)
        % \pause
  % \item Bijection with $\AgdaDatatype{Fin}~n$, for some $n : \mathbb{N}$? \\
        % (No extensionality, and we only get decidable equality with function extensionality.)
        % \pause
  % \item The \emph{mere existence} (propositional trucation) of one of the above? \\
        % (No decidable equality.)
        % \pause
  \item A higher-inductive type? \\
        (Works, but restricts us to HoTT. Also, HITs can be challenging. cf. Choudhury \& Fiore, 2023.)
\end{itemize}

\end{block}

  % Notice: A constructive witness of finite support for some dec. subset that preserves extensionality $S \to 2$ is exactly the type we were trying to define in the first place.

\begin{center}
Our approach: an enumeration, but using a \emph{sorted-by-construction} list.
\end{center}
\end{frame}


\begin{frame}{The Equational Theory of Finite Sets}
We ought to have:
\begin{itemize}
\item $\cup$, a binary operation which is: associative, commutative, idempotent, and unital (with $\emptyset$).
\item $\cap$, a binary operation which is: associative, commutative, idempotent, and distributes with $\cup$.
\item $\subseteq$, a binary relation which forms a lattice with $\cup$ and $\cap$.
\end{itemize}

\begin{block}{Theorem (In Set Theory) (Folklore?)}
  The free idempotent commutative monoid over a set $X$ is given by its finite powerset, $\mathcal{P}_{f}(X)$.
\end{block}

%\pause
%\begin{center}
%This gives us a way to evaluate the correctness of our proposed solutions for sets and multisets; they should satisfy the above universal properties.
%\end{center}
\end{frame}


\begin{frame}{Fresh Lists}
  \snippetdatafreshlist{}

  Originally due to Catarina Coquand.
  Generalisation to an arbitrary $R$ due to the Agda standard library.
\end{frame}


\begin{frame}{Sorted Lists}

\begin{itemize}
  \item To instantiate fresh lists as sorted lists, we set the relation to some irreflexive total order, $<$.

  \item The ordering ensures that any two lists with the same members are equal (extensionality!).

  \item Irreflexivity forces any given element to appear exactly once in any given list.
\end{itemize}

\begin{block}{``Definition'': Union (AKA Merge Sort)}
\begin{itemize}
\item $X \cup \emptyset = \emptyset \cup X = X$
  \item If both sides are non-empty, then compare the heads.
  \item Take the smaller to be the new head, and make a recursive call on the remainder for the tail. (If equal, drop one.)
  \item \ldots{}and discharge all the proof obligations in each case\ldots{}
\end{itemize}
\end{block}
\end{frame}


\begin{frame}{Extensionality Principle}
% We need to show that \union~satisfies the idem. comm. monoid laws.
% I redacted the definition of \union, so imagine how awful these proofs must be!

\begin{block}{Theorem: The Extensionality Principle for Sorted Lists}
  For all $xs, ys : \AgdaDatatype{FList}(X,<)$:

  \begin{center}
    $xs = ys$ iff $(a \in xs) \iff (a \in ys)$ for all $a : X$.
  \end{center}
\end{block}

This follows by appeal to decidability.
\end{frame}


\begin{frame}{The Categories at Issue}
\begin{center}
  We need ordering data to form the type $\AgdaDatatype{FList}(X,<)$.
\end{center}

\begin{block}{The Category \AgdaDatatype{STO}}
\begin{itemize}
  \item Objects: Sets, equipped with strict total orders.
  \item Morphisms: \emph{Not necessarily monotone} functions on the underlying sets.
\end{itemize}
\end{block}

\begin{block}{The Category \AgdaDatatype{OICMon}}
\begin{itemize}
  \item Objects: Idempotent commutative monoids, with strict total orders.
  \item Morphisms: \emph{Not necessarily monotone} monoid morphisms.
\end{itemize}
\end{block}
\end{frame}


\begin{frame}{A Free-Forgetful Adjunction}
\begin{block}{Theorem: The Universal Property of Ordered Idem. Comm. Monoids}
   $\AgdaDatatype{SList} : \AgdaDatatype{STO} \to \AgdaDatatype{OICMon}$ forms a functor which is left adjoint to the forgetful functor $\AgdaDatatype{\ensuremath{\mathcal{u}}} : \AgdaDatatype{OICMon} \to \AgdaDatatype{STO}$ defined by $\AgdaDatatype{\ensuremath{\mathcal{u}}}(X,<,\cdot,\epsilon) \coloneq (X,<)$.
 \end{block}
%We prove the adjunction via natural bijection of homsets. That is, we must show:
%
%\begin{center}
  %$\mathsf{Hom}_{OICMon}(\AgdaDatatype{FList}(A),~B) \cong \mathsf{Hom}_{STO}(A,~ \AgdaDatatype{\ensuremath{\mathcal{u}}}(B))$
%\end{center}
%\pause
%
%The hard direction involves showing that
%\begin{center}
%$\lambda~(f : \AgdaField{Carrier}~A \to \AgdaField{Carrier}~B)~\to~(\AgdaFunction{foldr}~(\lambda~a~b~\to~(f~a) \cdot b)~\epsilon)$
%\end{center}
%is a morphism of \AgdaDatatype{OICMon}.
\end{frame}

\begin{frame}{Multisets}

\begin{itemize}
  \item Setting the freshness relation to be some decidable (reflexive) total order $\leq$ yields sorted lists that allow duplicate elements.
  These realise finite multisets.
  \item $\in$ is valued in prop-valued for finite sets, but set-valued for finite multisets. So we need a different extensionality principle:
\end{itemize}

  \begin{block}{Theorem: Extensionality Principle for $\AgdaDatatype{FList}(X, \leq)$}
    For all $a : X$, and $xs, ys : \AgdaDatatype{FList}(X, \leq)$:
  \begin{itemize}
    \item Each $\AgdaDatatype{FList}(X, \leq)$ induces a ``multiplicity function'', $\AgdaFunction{count} : \AgdaDatatype{FList}(X,\leq) \to X \to \mathbb{N}$, such that \\
          $\AgdaFunction{count}~xs~a~=~\AgdaFunction{count}~ys~a$ iff $(a \in xs) \cong (a \in ys) $.
    \item $(a \in xs) \cong (a \in ys)$ iff $xs = ys$.
  \end{itemize}
 \end{block}
\end{frame}

\begin{frame}{Multisets}

 \begin{block}{Theorem: Universal Property of Ordered Commutative Monoids}
   $\AgdaDatatype{SListD} : \AgdaDatatype{DTO} \to \AgdaDatatype{OCMon}$ forms a functor which is left adjoint to the forgetful functor $\AgdaDatatype{\ensuremath{\mathcal{u}}} : \AgdaDatatype{OCMon} \to \AgdaDatatype{DTO}$.
 \end{block}

 Where DTO and OCMon are analogous to STO and OICMon from before, but without irreflexivity or idempotency.

\end{frame}


\begin{frame}{More Free Algebraic Structures}
  Different notions of ``freshness'' yield different free algebraic structures.

\begin{center}
\begin{tabular}{  |c|m{7em}|m{12em}| }
  \hline
  Freshness Relation & \centering{Data Structure} & \centering{Free Algebraic Structure} \tabularnewline
  \hline
  $\leq$, a total order & Sorted lists & Ordered comm. monoid \\
  $<$, a strict total order & Sorted lists w/o duplicates & Ordered idempotent comm.\ monoid \\
  $\lambda x. \lambda y. \bot$ & \AgdaDatatype{Maybe} & Pointed set \\
  $\lambda x. \lambda y. \top$ & \AgdaDatatype{List} & Monoid \\
  $\neq$ & Lists w/o duplicates & Left-regular band monoid \\
  $=$ & $1 + (A \times \mathbb{N}^{>0})$ & Reflexive partial monoid \\
  \hline
\end{tabular}
\end{center}
\end{frame}

\begin{frame}{Summary}

\begin{itemize}
  \item We saw the data type of (generalised) fresh lists.
  \item We saw how they realise finite sets and multisets, and proved the relevant universal properties.
  \item We glimpsed the zoo of other free algebraic structures that can be represented this way.
\end{itemize}

 Read our stuff:
\begin{itemize}
  \item Kupke, C., Nordvall Forsberg, F., Watters, S.: \emph{A fresh look at commutativity: free algebraic structures via fresh lists}. In: APLAS '23. \\
        \url{https://doi.org/10.1007/978-981-99-8311-7_7}
  \item Our Agda formalisation: \url{https://seanwatters.uk/agda/fresh-lists}
\end{itemize}
\end{frame}

\begin{frame}{Bonus: Why No Monotonicity?}

A few reasons:
\begin{itemize}
\item It breaks the adjunction.
\item We get a (subjectively) more natural notion of functoriality without it.
\item It's an implementation detail.
\item Without it, we get a nice result relating our constructions back to classical finite (multi)sets:
\end{itemize}

 If only there was a ``free strict total order on a set'', then we could ignore the ordering data and obtain the genuine $\mathcal{P}_{f}$.
 But such a thing is a weak form of AC called the Ordering Principle, which implies LEM.
 However:

\begin{block}{Theorem}
Assuming OP, $\mathsf{Set} \cong \mathsf{STO}$, $\mathsf{OICMon} \cong \mathsf{ICMon}$, etc.
\end{block}
\end{frame}


\begin{frame}{References}
\begin{itemize}
  \item Coquand, C.: \emph{A formalised proof of the soundness and completeness of a simply
        typed lambda-calculus with explicit substitutions}. In: Higher Order Symbolic Computation, 2002. \\
        \url{https://doi.org/10.1023/A:1019964114625}
  \item Choudhury, V., Fiore, M.: \emph{Free commutative monoids in Homotopy Type Theory}. In: MFPS '22. \\
        \url{https://doi.org/10.46298/entics.10492}
\end{itemize}
\end{frame}

\end{document}
