\documentclass[aspectratio=169]{beamer}
\usetheme{Boadilla}
\setbeamertemplate{navigation symbols}{}
\usepackage{tikz}
\usepackage{amsmath}
\usepackage{unicode-math}
\usepackage{mathtools}
\usepackage[conor]{agda}

\setmathfont{XITS Math}
%%%%%%%%%%
% latex preamble
% (missing unicode chars)

\usepackage{newunicodechar}
\newunicodechar{∈}{\ensuremath{\mathnormal{\in}}}
\newunicodechar{≡}{\ensuremath{\mathnormal{\equiv}}}
\newunicodechar{∷}{\ensuremath{\mathnormal{\dblcolon}}}
\newunicodechar{⊤}{\ensuremath{\mathnormal{\top}}}
\newunicodechar{₀}{\ensuremath{\mathnormal{_0}}}
\newunicodechar{₁}{\ensuremath{\mathnormal{_1}}}
\newunicodechar{₂}{\ensuremath{\mathnormal{_2}}}
\newunicodechar{π}{\ensuremath{\mathnormal{π}}}
\newunicodechar{∀}{\ensuremath{\mathnormal{\forall}}}
\newunicodechar{ℕ}{\ensuremath{\mathbb{N}}}
\newunicodechar{μ}{\ensuremath{\mathnormal{\mu}}}
\newunicodechar{ϕ}{\ensuremath{\mathnormal{\varphi}}}
\newunicodechar{ψ}{\ensuremath{\mathnormal{\psi}}}
\newunicodechar{η}{\ensuremath{\mathnormal{\eta}}}
\newunicodechar{≈}{\ensuremath{\mathnormal{\approx}}}

%%%%%%%%%%
% agda preamble

\begin{code}[hide]%
\>[0]\<%
\\
\>[0]\AgdaKeyword{open}\AgdaSpace{}%
\AgdaKeyword{import}\AgdaSpace{}%
\AgdaModule{Data.Nat}\AgdaSpace{}%
\AgdaKeyword{hiding}\AgdaSpace{}%
\AgdaSymbol{(}\AgdaOperator{\AgdaFunction{\AgdaUnderscore{}≟\AgdaUnderscore{}}}\AgdaSymbol{)}\<%
\\
\>[0]\AgdaKeyword{open}\AgdaSpace{}%
\AgdaKeyword{import}\AgdaSpace{}%
\AgdaModule{Data.Fin}\AgdaSpace{}%
\AgdaKeyword{using}\AgdaSpace{}%
\AgdaSymbol{(}\AgdaDatatype{Fin}\AgdaSymbol{;}\AgdaSpace{}%
\AgdaInductiveConstructor{zero}\AgdaSymbol{;}\AgdaSpace{}%
\AgdaInductiveConstructor{suc}\AgdaSymbol{;}\AgdaSpace{}%
\AgdaOperator{\AgdaFunction{\AgdaUnderscore{}≟\AgdaUnderscore{}}}\AgdaSymbol{)}\AgdaSpace{}%
\AgdaKeyword{renaming}\AgdaSpace{}%
\AgdaSymbol{(}\AgdaFunction{inject₁}\AgdaSpace{}%
\AgdaSymbol{to}\AgdaSpace{}%
\AgdaFunction{fin-inject₁}\AgdaSymbol{)}\<%
\\
\>[0]\AgdaKeyword{open}\AgdaSpace{}%
\AgdaKeyword{import}\AgdaSpace{}%
\AgdaModule{Data.Product}\<%
\\
\>[0]\AgdaKeyword{open}\AgdaSpace{}%
\AgdaKeyword{import}\AgdaSpace{}%
\AgdaModule{Relation.Binary.PropositionalEquality}\<%
\\
\>[0]\AgdaKeyword{open}\AgdaSpace{}%
\AgdaKeyword{import}\AgdaSpace{}%
\AgdaModule{Relation.Nullary.Decidable}\<%
\\
%
\\[\AgdaEmptyExtraSkip]%
\>[0]\AgdaKeyword{data}\AgdaSpace{}%
\AgdaDatatype{Opη}\AgdaSpace{}%
\AgdaSymbol{:}\AgdaSpace{}%
\AgdaPrimitive{Set}\AgdaSpace{}%
\AgdaKeyword{where}\<%
\\
\>[0][@{}l@{\AgdaIndent{0}}]%
\>[2]\AgdaInductiveConstructor{mu}\AgdaSpace{}%
\AgdaInductiveConstructor{nu}\AgdaSpace{}%
\AgdaSymbol{:}\AgdaSpace{}%
\AgdaDatatype{Opη}\<%
\\
%
\\[\AgdaEmptyExtraSkip]%
\>[0]\AgdaKeyword{data}\AgdaSpace{}%
\AgdaDatatype{Op₀}\AgdaSpace{}%
\AgdaSymbol{(}\AgdaBound{At}\AgdaSpace{}%
\AgdaSymbol{:}\AgdaSpace{}%
\AgdaPrimitive{Set}\AgdaSymbol{)}\AgdaSpace{}%
\AgdaSymbol{:}\AgdaSpace{}%
\AgdaPrimitive{Set}\AgdaSpace{}%
\AgdaKeyword{where}\<%
\\
\>[0][@{}l@{\AgdaIndent{0}}]%
\>[2]\AgdaInductiveConstructor{tt}\AgdaSpace{}%
\AgdaInductiveConstructor{ff}\AgdaSpace{}%
\AgdaSymbol{:}\AgdaSpace{}%
\AgdaDatatype{Op₀}\AgdaSpace{}%
\AgdaBound{At}\<%
\\
%
\>[2]\AgdaInductiveConstructor{at}\AgdaSpace{}%
\AgdaInductiveConstructor{¬at}\AgdaSpace{}%
\AgdaSymbol{:}\AgdaSpace{}%
\AgdaBound{At}\AgdaSpace{}%
\AgdaSymbol{→}\AgdaSpace{}%
\AgdaDatatype{Op₀}\AgdaSpace{}%
\AgdaBound{At}\<%
\\
%
\\[\AgdaEmptyExtraSkip]%
\>[0]\AgdaKeyword{data}\AgdaSpace{}%
\AgdaDatatype{Op₁}\AgdaSpace{}%
\AgdaSymbol{:}\AgdaSpace{}%
\AgdaPrimitive{Set}\AgdaSpace{}%
\AgdaKeyword{where}\<%
\\
\>[0][@{}l@{\AgdaIndent{0}}]%
\>[2]\AgdaInductiveConstructor{box}\AgdaSpace{}%
\AgdaInductiveConstructor{dia}\AgdaSpace{}%
\AgdaSymbol{:}\AgdaSpace{}%
\AgdaDatatype{Op₁}\<%
\\
%
\\[\AgdaEmptyExtraSkip]%
\>[0]\AgdaKeyword{data}\AgdaSpace{}%
\AgdaDatatype{Op₂}\AgdaSpace{}%
\AgdaSymbol{:}\AgdaSpace{}%
\AgdaPrimitive{Set}\AgdaSpace{}%
\AgdaKeyword{where}\<%
\\
\>[0][@{}l@{\AgdaIndent{0}}]%
\>[2]\AgdaInductiveConstructor{and}\AgdaSpace{}%
\AgdaInductiveConstructor{or}\AgdaSpace{}%
\AgdaSymbol{:}\AgdaSpace{}%
\AgdaDatatype{Op₂}\<%
\\
\>[0]\<%
\end{code}

%%%%%%%%%%
% agda snippets

\newcommand{\snippetwellscoped}{%
\begin{code}%
\>[0]\AgdaKeyword{data}\AgdaSpace{}%
\AgdaDatatype{WST}\AgdaSpace{}%
\AgdaSymbol{(}\AgdaBound{At}\AgdaSpace{}%
\AgdaSymbol{:}\AgdaSpace{}%
\AgdaPrimitive{Set}\AgdaSymbol{)}\AgdaSpace{}%
\AgdaSymbol{(}\AgdaBound{n}\AgdaSpace{}%
\AgdaSymbol{:}\AgdaSpace{}%
\AgdaDatatype{ℕ}\AgdaSymbol{)}\AgdaSpace{}%
\AgdaSymbol{:}\AgdaSpace{}%
\AgdaPrimitive{Set}\AgdaSpace{}%
\AgdaKeyword{where}\<%
\\
\>[0][@{}l@{\AgdaIndent{0}}]%
\>[2]\AgdaInductiveConstructor{tt}\AgdaSpace{}%
\AgdaInductiveConstructor{ff}%
\>[13]\AgdaSymbol{:}\AgdaSpace{}%
\AgdaDatatype{WST}\AgdaSpace{}%
\AgdaBound{At}\AgdaSpace{}%
\AgdaBound{n}\<%
\\
%
\>[2]\AgdaInductiveConstructor{at}\AgdaSpace{}%
\AgdaInductiveConstructor{¬at}%
\>[13]\AgdaSymbol{:}\AgdaSpace{}%
\AgdaBound{At}\AgdaSpace{}%
\AgdaSymbol{→}\AgdaSpace{}%
\AgdaDatatype{WST}\AgdaSpace{}%
\AgdaBound{At}\AgdaSpace{}%
\AgdaBound{n}\<%
\\
%
\>[2]\AgdaInductiveConstructor{and}\AgdaSpace{}%
\AgdaInductiveConstructor{or}%
\>[13]\AgdaSymbol{:}\AgdaSpace{}%
\AgdaSymbol{(}\AgdaBound{ϕ}\AgdaSpace{}%
\AgdaBound{ψ}\AgdaSpace{}%
\AgdaSymbol{:}\AgdaSpace{}%
\AgdaDatatype{WST}\AgdaSpace{}%
\AgdaBound{At}\AgdaSpace{}%
\AgdaBound{n}\AgdaSymbol{)}\AgdaSpace{}%
\AgdaSymbol{→}\AgdaSpace{}%
\AgdaDatatype{WST}\AgdaSpace{}%
\AgdaBound{At}\AgdaSpace{}%
\AgdaBound{n}\<%
\\
%
\>[2]\AgdaInductiveConstructor{box}\AgdaSpace{}%
\AgdaInductiveConstructor{dia}\AgdaSpace{}%
\AgdaSymbol{:}\AgdaSpace{}%
\AgdaSymbol{(}\AgdaBound{ϕ}\AgdaSpace{}%
\AgdaSymbol{:}\AgdaSpace{}%
\AgdaDatatype{WST}\AgdaSpace{}%
\AgdaBound{At}\AgdaSpace{}%
\AgdaBound{n}\AgdaSymbol{)}\AgdaSpace{}%
\AgdaSymbol{→}\AgdaSpace{}%
\AgdaDatatype{WST}\AgdaSpace{}%
\AgdaBound{At}\AgdaSpace{}%
\AgdaBound{n}\<%
\\
%
\>[2]\AgdaInductiveConstructor{mu}\AgdaSpace{}%
\AgdaInductiveConstructor{nu}%
\>[13]\AgdaSymbol{:}\AgdaSpace{}%
\AgdaSymbol{(}\AgdaBound{ϕ}\AgdaSpace{}%
\AgdaSymbol{:}\AgdaSpace{}%
\AgdaDatatype{WST}\AgdaSpace{}%
\AgdaBound{At}\AgdaSpace{}%
\AgdaSymbol{(}\AgdaInductiveConstructor{suc}\AgdaSpace{}%
\AgdaBound{n}\AgdaSymbol{))}\AgdaSpace{}%
\AgdaSymbol{→}\AgdaSpace{}%
\AgdaDatatype{WST}\AgdaSpace{}%
\AgdaBound{At}\AgdaSpace{}%
\AgdaBound{n}\<%
\\
%
\>[2]\AgdaInductiveConstructor{var}%
\>[13]\AgdaSymbol{:}\AgdaSpace{}%
\AgdaDatatype{Fin}\AgdaSpace{}%
\AgdaBound{n}\AgdaSpace{}%
\AgdaSymbol{→}\AgdaSpace{}%
\AgdaDatatype{WST}\AgdaSpace{}%
\AgdaBound{At}\AgdaSpace{}%
\AgdaBound{n}\<%
\end{code}}

\begin{code}[hide]%
\>[0]\AgdaKeyword{data}\AgdaSpace{}%
\AgdaDatatype{IsFP}\AgdaSpace{}%
\AgdaSymbol{\{}\AgdaBound{At}\AgdaSpace{}%
\AgdaSymbol{:}\AgdaSpace{}%
\AgdaPrimitive{Set}\AgdaSymbol{\}}\AgdaSpace{}%
\AgdaSymbol{\{}\AgdaBound{n}\AgdaSpace{}%
\AgdaSymbol{:}\AgdaSpace{}%
\AgdaDatatype{ℕ}\AgdaSymbol{\}}\AgdaSpace{}%
\AgdaSymbol{:}\AgdaSpace{}%
\AgdaDatatype{WST}\AgdaSpace{}%
\AgdaBound{At}\AgdaSpace{}%
\AgdaBound{n}\AgdaSpace{}%
\AgdaSymbol{→}\AgdaSpace{}%
\AgdaPrimitive{Set}\AgdaSpace{}%
\AgdaKeyword{where}\<%
\\
\>[0][@{}l@{\AgdaIndent{0}}]%
\>[2]\AgdaInductiveConstructor{mu}\AgdaSpace{}%
\AgdaSymbol{:}\AgdaSpace{}%
\AgdaSymbol{(}\AgdaBound{ϕ}\AgdaSpace{}%
\AgdaSymbol{:}\AgdaSpace{}%
\AgdaDatatype{WST}\AgdaSpace{}%
\AgdaBound{At}\AgdaSpace{}%
\AgdaSymbol{(}\AgdaInductiveConstructor{suc}\AgdaSpace{}%
\AgdaBound{n}\AgdaSymbol{))}\AgdaSpace{}%
\AgdaSymbol{→}\AgdaSpace{}%
\AgdaDatatype{IsFP}\AgdaSpace{}%
\AgdaSymbol{(}\AgdaInductiveConstructor{mu}\AgdaSpace{}%
\AgdaBound{ϕ}\AgdaSymbol{)}\<%
\\
%
\>[2]\AgdaInductiveConstructor{nu}\AgdaSpace{}%
\AgdaSymbol{:}\AgdaSpace{}%
\AgdaSymbol{(}\AgdaBound{ϕ}\AgdaSpace{}%
\AgdaSymbol{:}\AgdaSpace{}%
\AgdaDatatype{WST}\AgdaSpace{}%
\AgdaBound{At}\AgdaSpace{}%
\AgdaSymbol{(}\AgdaInductiveConstructor{suc}\AgdaSpace{}%
\AgdaBound{n}\AgdaSymbol{))}\AgdaSpace{}%
\AgdaSymbol{→}\AgdaSpace{}%
\AgdaDatatype{IsFP}\AgdaSpace{}%
\AgdaSymbol{(}\AgdaInductiveConstructor{nu}\AgdaSpace{}%
\AgdaBound{ϕ}\AgdaSymbol{)}\<%
\end{code}

\newcommand{\snippetscope}{%
\begin{code}%
\>[0]\AgdaKeyword{data}\AgdaSpace{}%
\AgdaDatatype{Scope}\AgdaSpace{}%
\AgdaSymbol{(}\AgdaBound{At}\AgdaSpace{}%
\AgdaSymbol{:}\AgdaSpace{}%
\AgdaPrimitive{Set}\AgdaSymbol{)}\AgdaSpace{}%
\AgdaSymbol{:}\AgdaSpace{}%
\AgdaDatatype{ℕ}\AgdaSpace{}%
\AgdaSymbol{→}\AgdaSpace{}%
\AgdaPrimitive{Set}\AgdaSpace{}%
\AgdaKeyword{where}\<%
\\
\>[0][@{}l@{\AgdaIndent{0}}]%
\>[2]\AgdaInductiveConstructor{[]}\AgdaSpace{}%
\AgdaSymbol{:}\AgdaSpace{}%
\AgdaDatatype{Scope}\AgdaSpace{}%
\AgdaBound{At}\AgdaSpace{}%
\AgdaInductiveConstructor{zero}\<%
\\
%
\>[2]\AgdaOperator{\AgdaInductiveConstructor{\AgdaUnderscore{}-,\AgdaUnderscore{}}}%
\>[167I]\AgdaSymbol{:}\AgdaSpace{}%
\AgdaSymbol{∀}\AgdaSpace{}%
\AgdaSymbol{\{}\AgdaBound{n}\AgdaSymbol{\}}\AgdaSpace{}%
\AgdaSymbol{(}\AgdaBound{Γ₀}\AgdaSpace{}%
\AgdaSymbol{:}\AgdaSpace{}%
\AgdaDatatype{Scope}\AgdaSpace{}%
\AgdaBound{At}\AgdaSpace{}%
\AgdaBound{n}\AgdaSymbol{)}\AgdaSpace{}%
\AgdaSymbol{\{}\AgdaBound{ϕ}\AgdaSpace{}%
\AgdaSymbol{:}\AgdaSpace{}%
\AgdaDatatype{WST}\AgdaSpace{}%
\AgdaBound{At}\AgdaSpace{}%
\AgdaBound{n}\AgdaSymbol{\}}\<%
\\
\>[.][@{}l@{}]\<[167I]%
\>[7]\AgdaSymbol{→}\AgdaSpace{}%
\AgdaSymbol{(}\AgdaBound{Γ₀}\AgdaSpace{}%
\AgdaSymbol{:}\AgdaSpace{}%
\AgdaDatatype{IsFP}\AgdaSpace{}%
\AgdaBound{ϕ}\AgdaSymbol{)}\AgdaSpace{}%
\AgdaSymbol{→}\AgdaSpace{}%
\AgdaDatatype{Scope}\AgdaSpace{}%
\AgdaBound{At}\AgdaSpace{}%
\AgdaSymbol{(}\AgdaInductiveConstructor{suc}\AgdaSpace{}%
\AgdaBound{n}\AgdaSymbol{)}\<%
\end{code}}

\newcommand{\snippetsublimelyscoped}{%
\begin{code}%
\>[0]\AgdaKeyword{mutual}\<%
\\
\>[0][@{}l@{\AgdaIndent{0}}]%
\>[2]\AgdaKeyword{data}\AgdaSpace{}%
\AgdaDatatype{SST}\AgdaSpace{}%
\AgdaSymbol{(}\AgdaBound{At}\AgdaSpace{}%
\AgdaSymbol{:}\AgdaSpace{}%
\AgdaPrimitive{Set}\AgdaSymbol{)}\AgdaSpace{}%
\AgdaSymbol{\{}\AgdaBound{n}\AgdaSpace{}%
\AgdaSymbol{:}\AgdaSpace{}%
\AgdaDatatype{ℕ}\AgdaSymbol{\}}\AgdaSpace{}%
\AgdaSymbol{(}\AgdaBound{Γ}\AgdaSpace{}%
\AgdaSymbol{:}\AgdaSpace{}%
\AgdaDatatype{Scope}\AgdaSpace{}%
\AgdaBound{At}\AgdaSpace{}%
\AgdaBound{n}\AgdaSymbol{)}\AgdaSpace{}%
\AgdaSymbol{:}\AgdaSpace{}%
\AgdaPrimitive{Set}\AgdaSpace{}%
\AgdaKeyword{where}\<%
\\
\>[2][@{}l@{\AgdaIndent{0}}]%
\>[4]\AgdaComment{--\ other\ constructors\ here...}\<%
\\
%
\>[4]\AgdaInductiveConstructor{var}\AgdaSpace{}%
\AgdaSymbol{:}\AgdaSpace{}%
\AgdaSymbol{(}\AgdaBound{x}\AgdaSpace{}%
\AgdaSymbol{:}\AgdaSpace{}%
\AgdaDatatype{Fin}\AgdaSpace{}%
\AgdaBound{n}\AgdaSymbol{)}\AgdaSpace{}%
\AgdaSymbol{→}\AgdaSpace{}%
\AgdaDatatype{SST}\AgdaSpace{}%
\AgdaBound{At}\AgdaSpace{}%
\AgdaBound{Γ}\<%
\\
%
\>[4]\AgdaInductiveConstructor{mu}%
\>[8]\AgdaSymbol{:}\AgdaSpace{}%
\AgdaSymbol{\{}\AgdaBound{ψ}\AgdaSpace{}%
\AgdaSymbol{:}\AgdaSpace{}%
\AgdaDatatype{WST}\AgdaSpace{}%
\AgdaBound{At}\AgdaSpace{}%
\AgdaSymbol{(}\AgdaInductiveConstructor{suc}\AgdaSpace{}%
\AgdaBound{n}\AgdaSymbol{)\}}\<%
\\
%
\>[8]\AgdaSymbol{→}\AgdaSpace{}%
\AgdaSymbol{(}\AgdaBound{ϕ}\AgdaSpace{}%
\AgdaSymbol{:}\AgdaSpace{}%
\AgdaDatatype{SST}\AgdaSpace{}%
\AgdaBound{At}\AgdaSpace{}%
\AgdaSymbol{(}\AgdaBound{Γ}\AgdaSpace{}%
\AgdaOperator{\AgdaInductiveConstructor{-,}}\AgdaSpace{}%
\AgdaInductiveConstructor{mu}\AgdaSpace{}%
\AgdaBound{ψ}\AgdaSymbol{))}\<%
\\
%
\>[8]\AgdaSymbol{→}\AgdaSpace{}%
\AgdaBound{ψ}\AgdaSpace{}%
\AgdaOperator{\AgdaDatatype{≈}}\AgdaSpace{}%
\AgdaBound{ϕ}\<%
\\
%
\>[8]\AgdaSymbol{→}\AgdaSpace{}%
\AgdaDatatype{SST}\AgdaSpace{}%
\AgdaBound{At}\AgdaSpace{}%
\AgdaBound{Γ}\<%
\\
%
\\[\AgdaEmptyExtraSkip]%
%
\>[2]\AgdaKeyword{data}\AgdaSpace{}%
\AgdaOperator{\AgdaDatatype{\AgdaUnderscore{}≈\AgdaUnderscore{}}}\AgdaSpace{}%
\AgdaSymbol{\{}\AgdaBound{At}\AgdaSpace{}%
\AgdaSymbol{:}\AgdaSpace{}%
\AgdaPrimitive{Set}\AgdaSymbol{\}}\AgdaSpace{}%
\AgdaSymbol{\{}\AgdaBound{n}\AgdaSpace{}%
\AgdaSymbol{:}\AgdaSpace{}%
\AgdaDatatype{ℕ}\AgdaSymbol{\}}\AgdaSpace{}%
\AgdaSymbol{\{}\AgdaBound{Γ}\AgdaSpace{}%
\AgdaSymbol{:}\AgdaSpace{}%
\AgdaDatatype{Scope}\AgdaSpace{}%
\AgdaBound{At}\AgdaSpace{}%
\AgdaBound{n}\AgdaSymbol{\}}\<%
\\
\>[2][@{}l@{\AgdaIndent{0}}]%
\>[4]\AgdaSymbol{:}\AgdaSpace{}%
\AgdaDatatype{WST}\AgdaSpace{}%
\AgdaBound{At}\AgdaSpace{}%
\AgdaBound{n}\AgdaSpace{}%
\AgdaSymbol{→}\AgdaSpace{}%
\AgdaDatatype{SST}\AgdaSpace{}%
\AgdaBound{At}\AgdaSpace{}%
\AgdaBound{Γ}\AgdaSpace{}%
\AgdaSymbol{→}\AgdaSpace{}%
\AgdaPrimitive{Set}\AgdaSpace{}%
\AgdaKeyword{where}\<%
\\
%
\>[4]\AgdaComment{--\ other\ constructors\ here...}\<%
\\
%
\>[4]\AgdaInductiveConstructor{var}\AgdaSpace{}%
\AgdaSymbol{:}\AgdaSpace{}%
\AgdaSymbol{(}\AgdaBound{x}\AgdaSpace{}%
\AgdaSymbol{:}\AgdaSpace{}%
\AgdaDatatype{Fin}\AgdaSpace{}%
\AgdaBound{n}\AgdaSymbol{)}\AgdaSpace{}%
\AgdaSymbol{→}\AgdaSpace{}%
\AgdaSymbol{(}\AgdaInductiveConstructor{var}\AgdaSpace{}%
\AgdaBound{x}\AgdaSymbol{)}\AgdaSpace{}%
\AgdaOperator{\AgdaDatatype{≈}}\AgdaSpace{}%
\AgdaSymbol{(}\AgdaInductiveConstructor{var}\AgdaSpace{}%
\AgdaBound{x}\AgdaSymbol{)}\<%
\\
%
\>[4]\AgdaInductiveConstructor{mu}%
\>[8]\AgdaSymbol{:}\AgdaSpace{}%
\AgdaSymbol{\{}\AgdaBound{ϕ}\AgdaSpace{}%
\AgdaSymbol{:}\AgdaSpace{}%
\AgdaDatatype{WST}\AgdaSpace{}%
\AgdaBound{At}\AgdaSpace{}%
\AgdaSymbol{(}\AgdaInductiveConstructor{suc}\AgdaSpace{}%
\AgdaBound{n}\AgdaSymbol{)\}}\<%
\\
%
\>[8]\AgdaSymbol{→}\AgdaSpace{}%
\AgdaSymbol{\{}\AgdaBound{ϕ'}\AgdaSpace{}%
\AgdaSymbol{:}\AgdaSpace{}%
\AgdaDatatype{SST}\AgdaSpace{}%
\AgdaBound{At}\AgdaSpace{}%
\AgdaSymbol{(}\AgdaBound{Γ}\AgdaSpace{}%
\AgdaOperator{\AgdaInductiveConstructor{-,}}\AgdaSpace{}%
\AgdaInductiveConstructor{mu}\AgdaSpace{}%
\AgdaBound{ϕ}\AgdaSymbol{)\}}\<%
\\
%
\>[8]\AgdaSymbol{→}\AgdaSpace{}%
\AgdaSymbol{(}\AgdaBound{p}\AgdaSpace{}%
\AgdaSymbol{:}\AgdaSpace{}%
\AgdaBound{ϕ}\AgdaSpace{}%
\AgdaOperator{\AgdaDatatype{≈}}\AgdaSpace{}%
\AgdaBound{ϕ'}\AgdaSymbol{)}\<%
\\
%
\>[8]\AgdaSymbol{→}\AgdaSpace{}%
\AgdaInductiveConstructor{mu}\AgdaSpace{}%
\AgdaBound{ϕ}\AgdaSpace{}%
\AgdaOperator{\AgdaDatatype{≈}}\AgdaSpace{}%
\AgdaInductiveConstructor{mu}\AgdaSpace{}%
\AgdaBound{ϕ'}\AgdaSpace{}%
\AgdaBound{p}\<%
\end{code}}
 % agda code snippets

% manual agdfa typesetting
\newcommand{\xvar}{\AgdaBound{$x$}}
\newcommand{\yvar}{\AgdaBound{$y$}}
\newcommand{\nil}{\AgdaField{$[]$}}
\newcommand{\cons}{\AgdaField{$\dblcolon$}}
\newcommand{\phitm}{\AgdaFunction{$\phi$}}
\newcommand{\datatm}{\AgdaDatatype{Tm}}
\newcommand{\abstm}{\AgdaField{abs}}
\newcommand{\union}{\AgdaFunction{$\cup$}}

\title[Extenensional Fin. Sets \& Multisets in TT]{Extensional Finite Sets and Multisets in Type Theory}
% \subtitle{Free Algebraic Structures via Fresh Lists}
\author[S. Watters]{Clemens Kupke \and Fredrik Nordvall Forsberg \and \underline{\textbf{Sean Watters}}}
\institute{University of Strathclyde}
\date{13/06/2024}

\begin{document}


\begin{frame}
  \titlepage{}
\end{frame}

\begin{frame}{Motivation}

  We want a data type for collections of unordered data (ie, finite sets and multisets), which:
  \begin{itemize}
          \smallskip
    \item Has decidable equality iff the underlying type does.
          \smallskip
    \item Satisfies the expected equational theories.
          \smallskip
    \item Works in ``standard'' MLTT.
  \end{itemize}

  %\begin{block}{Our Contribution}
  %\begin{itemize}
  %  \item We show that a suitable generalisation of Catarina Coquand's data type of fresh lists acheives the above, and also realises many other data structures via different instantiations.

  %  \item We prove that each instantiation satisfies the universal property of some free algebraic structure.

  %  \item In particular: the free pointed set, free monoid, free left-regular band monoid, and free reflexive partial monoid.
  %\end{itemize}
  %\end{block}

\end{frame}

% \begin{frame}{Motivation: A Normal Form for Contexts}
% \begin{exampleblock}{A Syntax with Binding}
% \snippetdatatm{}
% \end{exampleblock}
% \pause

% \begin{itemize}
%   \item  Consider the term $\abstm~\xvar~(\abstm~\yvar~\AgdaHole{?})$.
%         \pause

%   \item  Notice that the hole has type $\datatm~(\yvar~\cons~\xvar~\cons~\nil)$.
%         \pause

% \item  This means if we had some $\phitm : \datatm~(\xvar~\cons~\yvar~\cons~\nil)$, it would not fit!
% \end{itemize}
% \pause

% \begin{block}{Goal}
%   If we need such a $\phi$ to fit in such a hole, then we need contexts that are \emph{sets}, not lists.
% \end{block}
% \end{frame}


\begin{frame}{Notions of Subsets and Multisets in Type Theory}
% \begin{block}{In Short}
In short:
\begin{itemize}
          \smallskip
  \item Given some $S : \mathsf{Set}$, subsets of $S$ are unary predicates $S \to \mathsf{Prop}$.
          \smallskip
  \item \emph{Decidable} subsets are functions $S \to 2$.
          \smallskip
  \item Multisets over $S$ are functions $S \to \mathbb{N}$.
          \smallskip
\end{itemize}
% \end{block}
\bigskip

\pause{}

% \begin{block}{Desirable Properties}
Desirable properties:
\begin{itemize}
          \smallskip
  \item Extenensionality: \pause $(X = Y) \iff (\forall x.~x \in X \iff x \in Y)$
        \pause
          \smallskip
  \item Decidable Equality\pause{}, which we would expect to follow from finiteness.
          \smallskip
\end{itemize}
% \end{block}
\end{frame}

\begin{frame}{Finiteness, Decidable Equality, Extensionality}
  What approaches are available for finite subsets?
\begin{itemize}
          \smallskip
  \item  $S \to 2$? \\
        (No decidable equality.)
  \smallskip
  \item An enumeration list? \\
        (No extensionality.)
  \smallskip
        % \pause
  % \item Bijection with $\AgdaDatatype{Fin}~n$, for some $n : \mathbb{N}$? \\
        % (No extensionality, and we only get decidable equality with function extensionality.)
        % \pause
  % \item The \emph{mere existence} (propositional trucation) of one of the above? \\
        % (No decidable equality.)
        % \pause
  \item A higher-inductive type? (Choudhury \& Fiore, 2023; Joram \& Veltri, 2023)\\
        (Works, but restricts us to HoTT.)
  \smallskip
  \item A sorted list? \\
        (Our approach; need to treat the ordering data with care. See also: Appel \& Leroy, 2023; Krebbers, 2023; the Rocq libraries $\mathtt{fset}$, $\mathtt{extructures}$, $\mathtt{finmap}$, $\mathtt{ssrmisc}$. )
\end{itemize}
\end{frame}


\begin{frame}{The Equational Theory of Finite Sets}

  We expect notions of union and empty set, satisfying:
\begin{itemize}
  \item $X \cup \emptyset = \emptyset \cup X = X$ (unit).
  \item $X \cup X = X$ (idempotency).
        \smallskip
  \item $X \cup Y = Y \cup X$ (commutativity).
        \smallskip
  \item $X \cup (Y \cup Z) = (X \cup Y) \cup Z$ (associativity).
\end{itemize}


% \item $\cup$, a binary operation which is: associative, commutative, idempotent, and unital (with $\emptyset$).
% \item $\cap$, a binary operation which is: associative, commutative, idempotent, and distributes with $\cup$ (both ways).
% \item $\subseteq$, a binary relation which forms a lattice with $\cup$ and $\cap$.
\bigskip

\begin{block}{Theorem (Folklore?)}
  In the context of set theory, the finite powerset $\mathcal{P}_{f}(X)$, is the free idempotent commutative monoid over the set $X$.
\end{block}

\bigskip

We show that the free idem. comm. monoid is realised in type theory by sorted lists.


%\pause
%\begin{center}
%This gives us a way to evaluate the correctness of our proposed solutions for sets and multisets; they should satisfy the above universal properties.
%\end{center}
\end{frame}


\begin{frame}{Fresh Lists}

  We study sorted lists as an instance of the following generalisation:

  \snippetdatafreshlist{}

  Originally due to Catarina Coquand.
  Generalisation to an arbitrary $R$ due to the Agda standard library.
\end{frame}


\begin{frame}{Sorted Lists}


  \bigskip

  Sorted lists (without duplicates) arise as fresh lists over an irreflexive total order $<$:
\begin{itemize}
        \smallskip
  \item The ordering ensures that any two lists with the same elements are equal.
        \smallskip

  \item Irreflexivity forces any given element to appear exactly once in any given list.
\end{itemize}

\bigskip

\begin{block}{Monoid Structure}
  \begin{itemize}
          \item Unit: The empty list.
    \item Multiplication: Merge sort (defined by recursion on the lists, using totality of $<$).
        \smallskip
  \end{itemize}
\end{block}
\end{frame}


\begin{frame}{Extensionality Principle}
  Proving that the laws of $\emptyset$ and $\union$ hold for sorted lists by induction is messy. Instead:

  \bigskip

\begin{block}{Theorem: The Extensionality Principle for Sorted Lists}
  For all $xs, ys : \AgdaDatatype{FList}(X,<)$:

  \begin{center}
    $xs = ys$ iff $(a \in xs) \iff (a \in ys)$ for all $a : X$.
  \end{center}
\end{block}

\bigskip

With this sledgehammer, the proofs of the equations for $\cup$ become much easier.

\bigskip

\begin{block}{Theorem}
  $(\AgdaDatatype{FList}(X,<),~\cup,~\mathsf{nil})$ is an idempotent commutative monoid.
\end{block}

\end{frame}


\begin{frame}{Freeness}
  Freeness is formulated as a universal property; sorted lists form a functor which is left adjoint to a forgetful functor.
  But what are the categories?

  \bigskip
  \pause

\begin{block}{The Category \AgdaDatatype{STO}}
\begin{itemize}
  \item Objects: Sets, equipped with strict total orders.
  \item Morphisms: \emph{Not necessarily monotone} functions on the underlying sets.
\end{itemize}
\end{block}

\begin{block}{The Category \AgdaDatatype{OICMon}}
\begin{itemize}
  \item Objects: Idempotent commutative monoids, with strict total orders.
  \item Morphisms: \emph{Not necessarily monotone} monoid morphisms.
\end{itemize}
\end{block}

\bigskip
\pause

  % We need ordering data to form the type $\AgdaDatatype{FList}(X,<)$, but this implementation detail should not restrict the morphisms.
\begin{block}{Theorem: The Universal Property of Ordered Idem. Comm. Monoids}
   $\AgdaDatatype{SList} : \AgdaDatatype{STO} \to \AgdaDatatype{OICMon}$ forms a functor which is left adjoint to the forgetful functor $\AgdaDatatype{\ensuremath{\mathcal{u}}} : \AgdaDatatype{OICMon} \to \AgdaDatatype{STO}$ defined by $\AgdaDatatype{\ensuremath{\mathcal{u}}}(X,<,\cdot,\epsilon) \coloneq (X,<)$.
 \end{block}
\end{frame}



\begin{frame}{Multisets}

  Fresh lists over a decidable reflexive total order $\leq$ realise finite multisets.

  \bigskip

  $\in$ is prop-valued for finite sets, but set-valued for finite multisets. So we need a different extensionality principle:

  \begin{block}{Theorem: Extensionality Principle for $\AgdaDatatype{FList}(X, \leq)$}
    There is a ``multiplicity function'', $\AgdaFunction{count} : \AgdaDatatype{FList}(X,\leq) \to X \to \mathbb{N}$,
    such that:

    \smallskip

    For all $a : X$, and $xs, ys : \AgdaDatatype{FList}(X, \leq)$,

    \[
      \AgdaFunction{count}~xs~a~=~\AgdaFunction{count}~ys~a \quad\Leftrightarrow\quad (a \in xs) \cong (a \in ys) \quad\Leftrightarrow\quad xs = ys.
    \]
 \end{block}

 \bigskip
 \pause

 Analagous to before:

 \begin{block}{Theorem: Universal Property of Ordered Commutative Monoids}
   $\AgdaDatatype{SListD} : \AgdaDatatype{DTO} \to \AgdaDatatype{OCMon}$ forms a functor which is left adjoint to the forgetful functor $\AgdaDatatype{\ensuremath{\mathcal{u}}} : \AgdaDatatype{OCMon} \to \AgdaDatatype{DTO}$.
 \end{block}


\end{frame}


\begin{frame}{More Free Algebraic Structures}
  Different notions of ``freshness'' yield different free algebraic structures:

  \bigskip

\begin{center}
  \begin{tabular}{  |c|m{15em}|m{11em}| }
    \hline
    Freshness Relation & \centering{Free Algebraic Structure} & \centering{Data Structure} \tabularnewline
                                                                \hline
                                                                $\leq$, a total order & Ordered Commutative Monoid & Sorted lists \\
    $<$, a strict total order & Ordered Idempotent Comm.\ Monoid & Sorted lists w/o duplicates \\
    $\lambda x. \lambda y. \bot$ & Pointed Set & \AgdaDatatype{Maybe} \\
    $\lambda x. \lambda y. \top$ & Monoid & \AgdaDatatype{List} \\
    $\neq$ & Left-Regular Band Monoid & Lists without duplicates \\
    $=$ & Reflexive Partial Monoid & $1 + (A \times \mathbb{N}^{>0})$ \\
    \hline
  \end{tabular}
\end{center}
\end{frame}

\begin{frame}{Summary}

\begin{itemize}
  \item We saw the data type of (generalised) fresh lists.
  \smallskip
  \item We saw how they realise finite sets and multisets, and proved the relevant universal properties.
  \smallskip
  \item We glimpsed the zoo of other free algebraic structures that can be represented this way.
\end{itemize}

\bigskip

 Further reading:
\begin{itemize}
  \item Kupke, C., Nordvall Forsberg, F., Watters, S.: \emph{A fresh look at commutativity: free algebraic structures via fresh lists}. In: APLAS '23. \\
  \smallskip
        \url{https://doi.org/10.1007/978-981-99-8311-7_7}
  \item Full Agda formalisation: \url{https://seanwatters.uk/agda/fresh-lists}
\end{itemize}
\end{frame}

\begin{frame}{Bonus: Why No Monotonicity?}

A few reasons:
\begin{itemize}
  \smallskip
\item It breaks the adjunction.
  \smallskip
\item We get a (subjectively) more natural notion of functoriality without it.
  \smallskip
\item It's an implementation detail.
  \smallskip
\item Without it, we get a nice result relating our constructions back to classical finite (multi)sets:
\end{itemize}
\bigskip

 If only there was a ``free strict total order on a set'', then we could ignore the ordering data and obtain the genuine $\mathcal{P}_{f}$.
 But such a thing is a weak form of AC called the Ordering Principle, which implies LEM.
 However:

\bigskip
\begin{block}{Theorem}
Assuming OP, $\mathsf{Set} \cong \mathsf{STO}$, $\mathsf{OICMon} \cong \mathsf{ICMon}$, etc.
\end{block}
\end{frame}


\begin{frame}{References}
\begin{itemize}
  \item Coquand, C.: \emph{A formalised proof of the soundness and completeness of a simply
        typed lambda-calculus with explicit substitutions}.\\
        \url{https://doi.org/10.1023/A:1019964114625}
  \smallskip
  \item Choudhury, V., Fiore, M.: \emph{Free commutative monoids in Homotopy Type Theory}.\\
        \url{https://doi.org/10.46298/entics.10492}
  \smallskip
  \item Joram, P., Veltri, N.: \emph{Constructive final semantics of finite bags}.\\
        \url{https://doi.org/10.4230/LIPIcs.ITP.2023.20}
  \smallskip
  \item Appel, A.W., Leroy, X.: \emph{Efficient extensional binary tries}.\\
        \url{https://doi.org/10.1007/s10817-022-09655-x}
\end{itemize}
\end{frame}

\end{document}
